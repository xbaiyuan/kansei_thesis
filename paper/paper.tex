\documentclass{sumiilab-paper}
%% uplatex を使う場合:
% \documentclass[uplatex]{sumiilab-paper}
\usepackage{amsmath,amssymb,amsfonts,ascmac,cases,bm}
\usepackage{cite}
\usepackage{enumitem}
\usepackage{bussproofs}

%% ===============================================
%% 論文の表紙に表示される情報
%% ===============================================

% 論文の年度と種類
\paper{平成 n 年度 卒業論文}% 学部生
%\paper{平成 n 年度 修士論文}% 修士

% 論文のタイトル
\title{住井研究室の\\ステキな論文クラスファイルの使用例}

% 学籍番号と著者のお名前
\author{X0XX1234 ラムダ 小太郎}

% 著者の所属
\institute{東北大学 工学部\\電気情報物理工学科}% 学部生
%\institute{東北大学 大学院情報科学研究科\\情報基礎科学専攻}% 修士

% 指導教員のお名前
\supervisor{住井 英二郎 教授}% 指導教員
\subsupervisor{松田 一孝 准教授}% 論文指導教員(省略可)

% 論文発表日時
\date{平成 n 年1月1日 \quad 23:00--23:30}
% 発表場所
\venue{電子情報システム・応物系1号館2階トイレ}

%% ===============================================
%% ソースコードの設定
%% ===============================================

% プログラミング言語と表示するフォント等の設定
\lstset{
  language={[Objective]Caml},% プログラミング言語
  basicstyle={\ttfamily\small},% ソースコードのテキストのスタイル
  keywordstyle={\bfseries},% 予約語等のキーワードのスタイル
  commentstyle={},% コメントのスタイル
  stringstyle={},% 文字列のスタイル
  frame=trlb,% ソースコードの枠線の設定 (none だと非表示)
  numbers=left,% 行番号の表示 (none だと非表示)
  numberstyle={\footnotesize},% 行番号のスタイル
  xleftmargin=15pt,% 左余白
  xrightmargin=5pt,% 右余白
  keepspaces=true,% 空白を維持する
  mathescape,% $ で囲った部分を数式として表示する ($ がソースコード中で使えなくなるので注意)
  % 手動強調表示の設定
  moredelim=[is][\bfseries]{@*}{*@},
  moredelim=[is][\itshape]{@/}{/@}
}

%% ===============================================
%% 論文中で使う記号とかのマクロ定義
%% ===============================================

%% 論文中で繰り返し使う記号は次のように「マクロ」として実装しておくと良い。
%% TeX ソース中で \BOOL と書くと、\texttt{Bool} に置き換えてくれる。
%% フォントを変え忘れたりするリスクが減るし、あとから記号を変更するのも楽になる。

\newcommand{\keyword}[1]{\mathbf{#1}}
\newcommand{\BOOL}{\keyword{Bool}}
\newcommand{\TRUE}{\keyword{true}}
\newcommand{\FALSE}{\keyword{false}}
\newcommand{\IF}{\keyword{if}}
\newcommand{\THEN}{\keyword{then}}
\newcommand{\ELSE}{\keyword{else}}

\begin{document}
\maketitle

\begin{abstract}
ステキな論文の概要
\end{abstract}

%% 目次
\tableofcontents

%% ここから本文

\chapter{序論}

%% 参考文献は \cite{ID} とします(ID は refs.bib 内で文献につけた識別子)
%% BibTeX の使い方などは各自調べて下さい。
序論とか本論とか結論とか \cite{Pierce:TypeSystems}

\chapter{本論}

\section{BNF の書き方の例}

本節では、BNF によるプログラミング言語の構文の書き方を紹介する。
構文木の書き方は一つというわけではないので、幾つかのバリエーションを紹介する。
どの方法が良いと思うかは、個人の好みに依るところなので、好きなものを使えば良いと思う。

まず、次の方法では、array 環境を使って、BNF を書いている。
array 環境は数式環境中で表のようなものを書くときに使う。
基本的に、table 環境と使い方は同じである。
\[
%% 空白を明示的に開けるときは "\," "\ " "~" "\quad" "\qquad" などを使う。
%% 空白の幅は "\qquad" > "\quad" > "~" = "\ " > "\," の順で大きい。
%% "~" と "\ " は空白の代わりに改行を許すかどうかの違い("\ " だと改行される可能性がある)
\begin{array}{rl@{\qquad\qquad}r}
  t ::=
  & & \mbox{terms:} \\
  & x & \mbox{variables} \\
  & \lambda x.~t & \mbox{lambda abstraction} \\
  & t_1~t_2 & \mbox{application} \\
  & \TRUE & \mbox{true} \\
  & \FALSE & \mbox{false} \\
  & \IF~t_1~\THEN~t_2~\ELSE~t_3 & \mbox{if statement}
\end{array}
\]

他にも、次のように、align 環境を使っても、似たようなものを書くことができる。
\begin{align}
  t ::=
  & \tag*{terms:} \\
  & x \tag*{variables} \\
  & \lambda x.~t \tag*{lambda abstraction} \\
  & t_1~t_2 \tag*{application} \\
  & \TRUE \tag*{true} \\
  & \FALSE \tag*{false} \\
  & \IF~t_1~\THEN~t_2~\ELSE~t_3 \tag*{if statement}
\end{align}
array 環境を愚直に使う場合と比べて、式が中央揃えになるという点と、
``variables'' とかの説明が右端に来ている点が違う。
説明は tag* マクロで出しており、これはもともと式番号を指定するためのものなので、
若干使い方がおかしい気もするが、まぁ、いいだろう。
自分の好みの方を使うと良いだろう。

BNF 全体を左揃えにしたいならば、次のように、flalign 環境を使うと良い。
align 環境と違って、\verb|&| を余分に1つ付ける必要がある、ということに注意して欲しい(詳しくはソースコードを見よ)。
\begin{flalign}
  t ::=
  & & \tag*{terms:} \\ % & を余分に1つ付けること!
  & x \tag*{variables} \\
  & \lambda x.~t \tag*{lambda abstraction} \\
  & t_1~t_2 \tag*{application} \\
  & \TRUE \tag*{true} \\
  & \FALSE \tag*{false} \\
  & \IF~t_1~\THEN~t_2~\ELSE~t_3 \tag*{if statement}
\end{flalign}

\section{導出木の書き方の例}

導出木の書き方も色々あるが、ここでは、bussproofs.sty を使った方法を紹介する。
導出木は、手書きでも書きにくいが、\LaTeX だから書きやすいというわけでもなく、
(使うパッケージにも依るが)そこそこの苦労は必要である。
bussproofs.sty を除く多くの方法では、frac などをベースに「分数」で導出木を書く。
bussproofs.sty はこれらとは全く異なるインタフェースであり、慣れれば比較的解りやすい。
bussproofs.sty の動作は、(導出木を要素とする)スタックをイメージすると解りやすい。
よく使うマクロは次の通り。
\begin{itemize}
\item \lstinline|\AxiomC{...}|:Axiom を push する(導出木では葉に相当)
\item \lstinline|\UnaryInfC{...}|:スタックから部分導出木(仮定)を1つ pop して、
  それを新たに作ったノード(結論)の子供にすることで、新たな部分導出木を作成し、push する。
\item \lstinline|\BinaryInfC{...}|:スタックから部分導出木(仮定)を2つ pop して、
  \lstinline|\UnaryInfC| と同様の動作を行う。
\item \lstinline|\TrinaryInfC{...}|:スタックから部分導出木(仮定)を3つ pop して、
  \lstinline|\UnaryInfC| と同様の動作を行う。
\end{itemize}

実際の使い方は以下の通り。

%% T-Var
\begin{prooftree}
  \AxiomC{$x:T \in \Gamma$}
  \RightLabel{\textsc{T-Var}}
  \UnaryInfC{$\Gamma \vdash x : T$}
\end{prooftree}
%% T-Abs
\begin{prooftree}
  \AxiomC{$\Gamma, x:T \vdash t : U$}
  \RightLabel{\textsc{T-Abs}}
  \UnaryInfC{$\Gamma \vdash \lambda x.~t : T \to U$}
\end{prooftree}
%% T-App
\begin{prooftree}
  \AxiomC{$\Gamma \vdash t_1 : T \to U$}
  \AxiomC{$\Gamma \vdash t_2 : T$}
  \RightLabel{\textsc{T-App}}
  \BinaryInfC{$\Gamma \vdash t_1~t_2 : U$}
\end{prooftree}

\begin{prooftree}
  \AxiomC{}
  \RightLabel{\textsc{T-True}}
  \UnaryInfC{$x : \BOOL \to \BOOL \vdash \TRUE : \BOOL$}
  \RightLabel{\textsc{T-Abs}}
  \UnaryInfC{$\vdash \lambda x.~\TRUE : (\BOOL \to \BOOL) \to \BOOL$}
  \AxiomC{$y : \BOOL \in y : \BOOL$}
  \RightLabel{\textsc{T-Var}}
  \UnaryInfC{$y : \BOOL \vdash y : \BOOL$}
  \RightLabel{\textsc{T-Abs}}
  \UnaryInfC{$\vdash \lambda y.~y : \BOOL \to \BOOL$}
  \RightLabel{\textsc{T-App}}
  \BinaryInfC{$\vdash (\lambda x.~\TRUE)~(\lambda y.~y) : \BOOL$}
\end{prooftree}

\section{定理環境}

この論文クラスファイルでは、デフォルトで以下の定理環境を提供している。

\begin{theorem}[定理のタイトル]
  定理の内容
\end{theorem}

\begin{lemma}[補題のタイトル]
  補題の内容
\end{lemma}

\begin{corollary}[系のタイトル]
  系の内容
\end{corollary}

\begin{proposition}[命題のタイトル]
  命題の内容
\end{proposition}

\begin{definition}[定義のタイトル]
  定義の内容
\end{definition}

\begin{example}[例のタイトル]
  例の内容
\end{example}

\begin{assumption}[仮定のタイトル]
  仮定の内容
\end{assumption}

\begin{axiom}[公理のタイトル]
  公理の内容
\end{axiom}

\begin{proof}[証明のタイトル]
  証明の内容 \qed
\end{proof}

\begin{proof*}[証明のタイトル]
  証明の内容(番号なしの証明環境。証明を \lstinline|\ref| で参照する必要がないなら、こっちを使うほうが自然かも) \qed
\end{proof*}

\subsection{定理環境の使い方の例}

\begin{lemma}
  \label{lem:interesting-lemma}
  論文の中で最重要とは言えないような性質・命題は補題 (lamma) にする。
  補題や定理から直ちに導けるような軽い命題は系 (corollary) にする(細かい使い分けは人による)。
\end{lemma}

\begin{proof*}
  \lstinline|proof*| のように、アスタリスク付きの環境では、番号が付かない。
\end{proof*}

\begin{theorem}
  \label{thm:wonderful-theorem}
  提案手法の最も重要な性質や命題は、定理 (theorem) として書く。
  読者の心をくすぐる興味深いステートメントを書こう。
\end{theorem}

\begin{proof*}
  定理 \ref{thm:wonderful-theorem} の華麗な証明。その美しい証明に、読者の目は釘付けだ!
  \begin{enumerate}[leftmargin=0pt,itemindent=*,label=Case \arabic*.]
  \item 自明
  \item 補題 \ref{lem:interesting-lemma} から直ちに導ける。
  \item 言うまでもない。目を瞑れば証明が見えてくる。
  \item あんまり自明じゃない
    \begin{enumerate}[label=(\roman*)]
    \item 自明じゃないと思ったけど、やっぱり自明だった
    \item ほらね、こんなに簡単
    \end{enumerate}
  \end{enumerate}
\end{proof*}

\section{ソースコード}

ソースコード\ref{src:listup_nodes}は二分木を深さ優先探索して、ノードを列挙する関数である。
\begin{lstlisting}[caption=二分木のノードのリストアップ,label=src:listup_nodes]
type 'a bin_tree =
  | Leaf of 'a
  | Node of 'a bin_tree * 'a bin_tree

let rec listup_nodes = function
  | Leaf x -> [x]
  | Node (r, l) -> (listup_nodes r) @ (listup_nodes l)
\end{lstlisting}
ソースコードの書き方等については slide ブランチの slide.tex を参照されたし。

\chapter{結論}

\chapter*{謝辞}

ステキな論文の謝辞

%% 参考文献: bibtex
\bibliographystyle{junsrt}
\bibliography{refs}

\end{document}
